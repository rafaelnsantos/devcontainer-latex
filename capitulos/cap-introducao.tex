%%%% CAPÍTULO 1 - INTRODUÇÃO
%%
%% Deve apresentar uma visão global da pesquisa, incluindo: breve histórico, importância e justificativa da escolha do tema,
%% delimitações do assunto, formulação de hipóteses e objetivos da pesquisa e estrutura do trabalho.

%% Título e rótulo de capítulo (rótulos não devem conter caracteres especiais, acentuados ou cedilha)
\chapter{Introdução}\label{cap:introducao}

Parte inicial do texto, na qual devem constar o tema e a delimitação do assunto tratado, objetivos da pesquisa e outros elementos necessários para situar o tema do trabalho. \textcolor{red}{Após o início de uma seção, recomenda-se a inserção de um texto ou, no mínimo, uma nota explicativa sobre a seção iniciada. Evitar, por exemplo:}

% Segundo \citeonline{Coulouris2013}.

% Segundo \citeonline[p. 40]{Coulouris2013}.

% Citação no final do Parágrafo~\cite{Coulouris2013}. 

% Citação no final do Parágrafo com número de página~\cite[p. 40]{Coulouris2013}.

% %(Modelo de referência: pessoa jurídica)
% Citação no final do Parágrafo~\cite{NBR6023:2018}

% %(Modelo de referência: pessoa jurídica)
% Citação no final do Parágrafo~\cite{NBR6027:2012}

% %(Modelo de referência: pessoa jurídica)
% Citação no final do Parágrafo~\cite{NBR6028:2021}

% Segundo a \citeonline{NBR14724:2011}.

% Citação no final do Parágrafo~\cite{NBR10520:2002}

% Citação no final do Parágrafo~\cite{NBR14724:2011}.

% % (Modelo de referência de trabalho acadêmico).
% Citação no final do Parágrafo~\cite{Andrade2005}

% % (Modelo de referência: capítulo de livro).
% Citação no final do Parágrafo~\cite{Borges2014}

% % (Modelo de referência: leis, decretos, portarias, etc.)
% Citação no final do Parágrafo~\cite{BRASIL:1998}

% % (Modelo de referência: livro com subtítulo). Nome com sufixo "Von" - Configuração no bib
% Citação no final do Parágrafo~\cite[p. 66]{KROGH:2001}

% Citação no final do Parágrafo~\cite{Faina2001}

% % (Modelo de referência: livro com subtítulo).
% Citação no final do Parágrafo~\cite{Davenport2012}

% % (Modelo de referência: artigo de periódico).
% Citação no final do Parágrafo~\cite{Monteiro2009}

% %(Modelo de referência: artigo de periódico). Nome familiar "Junior"
% Citação no final do Parágrafo~\cite{Sanches2024}

% % (Modelo de referência: trabalho publicado em evento).
% Citação no final do Parágrafo~\cite{Renaux2001}

\section{Contextualização}

\subsection{Memorial de Pesquisa}


\section{Paginação}

Todas as folhas do trabalho, a partir da folha de rosto, devem ser contadas sequencialmente, mas não numeradas. A numeração deve ser inserida à partir da primeira folha da parte textual (Introdução), em algarismos arábicos, no canto superior direito da folha. Havendo apêndices e anexos, as suas folhas devem ser paginadas de maneira contínua.

\section{Exemplos de utilização de numeração progressiva}

Nos títulos com indicativo numérico não se utilizam pontos, hífen, travessão, ou qualquer sinal após o indicativo de seção ou de título. 

A numeração progressiva para as seções do texto deve ser adotada para evidenciar a sistematização do conteúdo do trabalho. 

Destacam-se gradativamente os títulos das seções, utilizando-se tipograficamente com recursos como letra maiúscula, negrito, itálico ou sublinhado. 

No sumário, os títulos devem aparecer de forma idêntica ao texto.

\textcolor{red}{Ver os exemplos na folha seguinte:}

